\section*{Segmentation of Overlapping Macrophages}
% 
Cell migration is important in many biological processes, one of them
being the immune system, a network of cells that work together to protect the 
body. Macrophages, a type of white blood cell, settle in lymphoid tissues and 
the liver, thus trapping foreign particles\cite{Martinez2008}. 
An unbalanced migratory result from these cells results in human disease, 
like cancer or autoimmune. Study of migration of the aforementioned cells 
becomes relevant because cell migration is essential for the 
homeostasis in adults\cite{pocha2014}. 
Life scientists have used model organisms to understand human cell processes. 
Such organism is the \emph{Drosophila melanogaster}, the common fruit fly, 
which has shown to offer insights about how macrophages integrate the signals
from the environment and other cells into migration\cite{wood2017}.\medskip\\
% 
The research project involves developing computer algorithms to analyse the 
movement of macrophages taken from the model organism 
\emph{Drosophila melanogaster}, which were genetically modified to fluoresce 
when looked under a microscope\cite{Stramer2010}. While the process of acquiring 
these sequences of images containing macrophages has a lot of work behind it, 
the images still suffer from artefacts and noise which makes it hard to 
make out the cells from the background, even for the human eye. 
Moreover, the interactions between the cells -which are crucial for scientists
to analyse- cause traditional algorithms\cite{Henry2013,lu2015,Caselles} 
to fail to distinguish one cell from
another, only detecting the cluster of cells as one single entity, which is 
called \emph{clump} in some literature\cite{lu2015}. The presence of clumps
creates the need for an algorithm capable to tell the cells apart, even when 
they overlap.\medskip\\
% 
Preliminary results in the study of movement of migrating macrophages involve the 
analysis of a single frame, where some cells overlap and other cells are 
isolated from the clusters, or \emph{clumps} of cells. When detecting a clump, 
an algorithm estimates the position of the intersection between two or more
overlapping cells. These points improve on other corner-finding algorithms, like 
Harris'\cite{Harris1988}. From the intersection points, an analysis was made of 
the boundaries adjacent to these points and the portion of the cell hidden 
from segmentation due to overlapping was found in a large number of cases. The 
algorithm to find intersection points was proven to work with a variety of 
synthetic data, and has proven consistent with clumps in real macrophages 
images. Although the methods of finding the overlapped portions of the 
cells are still in development. However, current experimental results 
demonstrate the promise of this approach 
to produce correct segmentations of overlapping cells.