\section*{Segmentation of Overlapping Macrophages}
%
Cell migration is an important process as it is present in many events of
disease like cancer metastasis and wound healing.
The main objective of this work is to analyse and characterise cell
migration, in this case by following macrophages in a model organism.
Study of migration of the aforementioned cells is relevant because
cell migration is essential for the homeostasis in adults\cite{pocha2014},
where an unbalanced migratory response from these cells results in
human disease, like cancer or autoimmune.
Life scientists have used model organisms to understand human cell processes.
The model organism \emph{Drosophila melanogaster}, the common fruit fly,
has shown insights about how macrophages integrate the signals
from the environment and other cells into migration \cite{wood2017}.
The organisms have been genetically modified to fluoresce
when looked under a microscope, where both the nuclei and
microtubules are shown \cite{Stramer2010}.
\medskip\\
%
This work involves the developing of computer algorithms to analyse the
movement of macro\-phages taken from the model organism
\emph{Drosophila melanogaster}. While a lot of development
has gone into acquiring sequences of images containing macrophages,
the images still suffer from artefacts and noise which makes it hard to
make out the cells from the background, even for the human eye.
Furthermore, the interactions between the cells
-which are crucial for scientists to analyse-
cause traditional algorithms\cite{Henry2013,lu2015,Caselles}
to create an ambiguity between the interacting cells, only detecting them
as one single entity, or \emph{clump}.
The presence of clumps creates the need for an algorithm capable to solve
the ambiguity generated by the overlapping of cells.
\medskip\\
%
Results of overlapping macrophages in a single frame will be
presented, where some cells overlap and other cells are
isolated from the \emph{clumps} of cells. When detecting a clump,
the algorithm estimates the position of the \emph{junctions} or intersections
between two or more overlapping cells. From these points, two techniques
to disambiguate the overlapped section of the cells will be discussed.
Finally, preliminary results of the incorporation of temporal components
into the segmentation will be presented. Once having analysed a single
frame, detected cells' shapes could be exploited as priors for the
shape of the same cell in subsequent frames, thus helping to acquire a
more robust segmentation that would lead to analysis in the shape and the
movement of migrating macrophages.
