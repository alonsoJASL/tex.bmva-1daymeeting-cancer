  \section*{Segmentation of Overlapping Macrophages}
%
Cell migration is important in many biological processes, one of them
being the immune system, a network of cells that work together to protect the
body. Macrophages, a type of white blood cell, settle in lymphoid tissues and
the liver, thus trapping foreign particles\cite{Martinez2008}.
Study of migration of the aforementioned cells
becomes relevant because cell migration is essential for the
homeostasis in adults\cite{pocha2014}, where
an unbalanced migratory response from these cells results in human disease,
like cancer or autoimmune.
Life scientists have used model organisms to understand human cell processes.
The model organism \emph{Drosophila melanogaster}, the common fruit fly,
has shown insights about how macrophages integrate the signals
from the environment and other cells into migration\cite{wood2017}.
The organisms have been genetically modified to fluoresce
when looked under a microscope\cite{Stramer2010}\medskip\\
%
This talk will involve the developing of computer algorithms to analyse the
movement of macrophages taken from the model organism
\emph{Drosophila melanogaster}. While a lot of development
has gone into acquiring sequences of images containing macrophages,
the images still suffer from artefacts and noise which makes it hard to
make out the cells from the background, even for the human eye.
Moreover, the interactions between the cells -which are crucial for scientists
to analyse- cause traditional algorithms\cite{Henry2013,lu2015,Caselles}
to fail to distinguish one cell from
another, only detecting the cluster of cells as one single entity, referred to
as \emph{clump}. The presence of clumps
creates the need for an algorithm capable to tell the cells apart, even when
they overlap.\medskip\\
%
Results of overlapping macrophages in a single frame will be
presented, where some cells overlap and other cells are
isolated from the \emph{clumps} of cells. When detecting a clump,
an algorithm estimates the position of the \emph{junctions} or intersections
between two or more overlapping cells. From these points, two techniques
to disambiguate the overlapped section of the cells will be discussed.
Finally, preliminary results of the incorporation of temporal components
into the segmentation will be presented. 
